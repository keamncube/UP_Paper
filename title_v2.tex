\title{Regulation and bank lending in South Africa: A narrative index approach\thanks{The authors thank the South African Reserve Bank for their financial assistance. This paper is provided as part of the Bank's call for papers on the impact of prudential regulation on financial services. We also thank Sanelisiwe Hlatshwayo for invaluable research assistance.}}

\author {Xolani Sibande\footnote{Economic Research Department, South African Reserve Bank; Email: xolani.sibande@resbank.co.za.} \and
Dumakude Nxumalo\footnote{Department of Economics, University of Pretoria, Pretoria, South Africa; Email: dumakude.nxumalo@up.ac.za} \and
Keaoleboga Mncube\footnote{Department of Economics, University of Pretoria, Pretoria, South Africa; Email: keaolebogamncube@gmail.com} \and
Steve Koch\footnote{Department of Economics, University of Pretoria, Pretoria, South Africa; Email: steve.koch@up.ac.za} \and
Nicola Viegi\footnote{Department of Economics, University of Pretoria, Pretoria, South Africa; Email: viegin@gmail.com}}

\date{\today}
\maketitle

\begin{center}
\textbf{Abstract}
\end{center}

\begin{abstract}
This study estimates and contrasts the impact of potentially contradictory regulation on the bank lending rates and volumes of the South Africa's largest banks. The regulations we consider are macroprudential regulation that seek to achieve stability in the finance sector and finance regulations intended to achieve greater financial inclusion.  We estimate separate panel data regressions for the different types of regulations we consider. Our results suggest that macroprudential policy is working as intended as it is associated with increases in interest rates on unsecured lending rates, decreases in short-term secured and mortgage lending rates. We observe that lending growth rates in unsecured and secured credit increase and decrease in mortage lending. Inclusion focused regulation is associated with increased bank lending rates in unsecured credit and lower mortgage rates. We observe a decrease in the growth of unsecured lending for corporates and an increase in secured lending for corporates. Results suggest inclusion focused regulation is at odds with its objectives and is consistent with the impact of macroprudential policy on bank pricing.
\end{abstract}


\noindent\textbf{Keywords}: Bank lending, narrative methods, finance regulation\\
\textbf{JEL Codes}: G01, G18, G28, G32, G38
\newpage